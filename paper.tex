\documentclass[10pt]{article}

% amsmath package, useful for mathematical formulas
\usepackage{amsmath}
% amssymb package, useful for mathematical symbols
\usepackage{amssymb}

% graphicx package, useful for including eps and pdf graphics
% include graphics with the command \includegraphics
\usepackage{graphicx}

% cite package, to clean up citations in the main text. Do not remove.
\usepackage{cite}

\usepackage{color} 

% Use doublespacing - comment out for single spacing
\usepackage{setspace} 
\doublespacing


% Use the PLoS provided bibtex style
% \bibliographystyle{PLoS2009}
\bibliographystyle{unsrt}

% Remove brackets from numbering in List of References
\makeatletter
\renewcommand{\@biblabel}[1]{\quad#1.}
\makeatother


% Leave date blank
\date{}

\pagestyle{myheadings}
%% ** EDIT HERE **
%% Please insert a running head of 30 characters or less.  
%% Include it twice, once between each set of braces
\newcommand\titlestring{Stochastic tree-adjoining grammars for modeling retrotransposons}
\newcommand\authorstring{
Lawrence Uricchio, Ian Holmes
}

\markboth{\titlestring}{\titlestring}


%% ** EDIT HERE **
%% PLEASE INCLUDE ALL MACROS BELOW

\usepackage{setspace}
\doublespacing



\usepackage{array}


% Labels & references for sections, figures and tables
% Comment out \secref and \seclabel for PLoS -- they don't have numbered section references.
\newcommand{\secref}[1]{Section~\ref{sec:#1}}
\newcommand{\seclabel}[1]{\label{sec:#1}}
\newcommand{\secname}[1]{``#1''}  % PLoS-style section names

% "Text S1", "Text S2", etc.
\newcommand{\supptext}[1]{Text S#1}

% "Dataset S1", "Dataset S2", etc.
\newcommand{\dataset}[1]{Dataset S#1}

% Appendix
\newcommand{\appref}[1]{Appendix~\ref{app:#1}}
\newcommand{\applabel}[1]{\label{app:#1}}

% Figure
\newcommand{\figref}[1]{Figure~\ref{fig:#1}}
\newcommand{\figlabel}[1]{\label{fig:#1}}

% Table
\newcommand{\tabnum}[1]{\ref{tab:#1}}
\newcommand{\tabref}[1]{Table~\tabnum{#1}}
\newcommand{\tablabel}[1]{\label{tab:#1}}

% Equation
\newcommand{\eqnref}[1]{Equation~\ref{eqn:#1}}
\newcommand{\eqnlabel}[1]{\label{eqn:#1}}


% need cite, check me, and other notes to self
\newcommand\needcite{{\bf [CITE]}}
\newcommand\checkme{{\bf [CHECK]}}



%% END MACROS SECTION

\begin{document}

% Title must be 150 words or less
\begin{flushleft}
  {\Large
    \textbf{\titlestring}
  }
\\
\authorstring
\end{flushleft}


% Please keep the abstract between 250 and 300 words
%\newpage
\section*{Abstract}
TAGs and parsers for biological repeats.

% Table of contents
\tableofcontents

% Please keep the Author Summary between 150 and 200 words
% Use first person. PLoS ONE authors please skip this step. 
% Author Summary not valid for PLoS ONE submissions.   
%\newpage
%\section{Author Summary}


\section{Introduction}

Transposable elements (TEs), or {\em transposons}, are of great interest in molecular evolution \cite{Feschotte2007}, and an important aspect of genome annotation.
There are several specializations in the overall task of transposon annotation:
PILER \cite{EdgarMyers2005} specializes in {\em de novo} transposon discovery,
while REPCLASS specializes in classification of found transposons \cite{FeschotteKeswaniEtAl2009}.

Many such programs classify TEs by their general structural features, particularly their terminal repeats:
LTRs (Long Terminal Repeats) and TIRs (Terminal Inverted Repeats).

It is useful to build databases and profiles of known transposon families, for the purpose of classifying new ones.
To date, the most comprehensive database of known transposons is REPBASE \cite{KapitonovJurka2008},
whose profiles rely only on primary sequence homology models;
that is, they do not make explicit use of terminal repeat structure.

A promising approach, that combines profile Hidden Markov Models (HMMs) of primary sequence homology (at the level of TE protein domains)
with fast algorithms for detecting LTRs,
is taken by LTRdigest \cite{pmid19786494}.
The purpose of this paper is to represent the hybrid modeling approach of LTRdigest using formal grammars.

\section{Definitions}

\subsection{Tree-Adjoining Grammars}

We define a minimal normal form of Tree-Adjoining Grammars (TAGs)
suited to biological sequence analysis,
as opposed to the linguistic representation elsewhere \cite{JoshiSchabes97}.
TAGs have previously been used in bioinformatics
to model pseudoknots and other RNA structures \cite{MatsuiEtAl2005,ChiangJoshiSearls2006}
and to model local duplications (Hickey and Blanchette, pers. comm.).

\newcommand\grammar{{\cal G}}
\newcommand\nodelabels{{\cal N}}
\newcommand\terminals{{\cal T}}
\newcommand\startsymbol{S}
\newcommand\rules{{\cal R}}
\newcommand\weight{{\cal W}}

A TAG is a tuple $\grammar=(\nodelabels,\terminals,\startsymbol,\rules,\weight)$ where
$\nodelabels$ is a set of {\em node labels},
$\terminals$ is a set of {\em terminals} (disjoint from $\nodelabels$),
$\startsymbol \in \nodelabels$ is a distinguished {\em start label},
$\rules$ is a set of {\em transformation rules}
and $\weight:\rules \to [0,\infty)$ is a {\em rule weight function}.

\newcommand\outseq{Z}
\newcommand\outsubseq[2]{\outseq[#1 \ldots #2]}

The process of generating an output sequence $\outseq \in \terminals^\ast$ using $\grammar$
is referred to as a {\em derivation} of $\outseq$.
The derivation consists of repeated local application of transformation rules to {\em intermediate trees}, beginning with the initial tree

$\startsymbol$


{\bf more to go here}


\subsection{A parsing algorithm}

We can define a general parsing algorithm for TAGs
that is the equivalent of the CYK (Cocke-Younger-Kasami) algorithm for SCFGs, as follows.

For the given output sequence $\outseq \in \terminals^\ast$, let $\outsubseq{i}{j+1}$ denote the substring from $i$ through $j$ inclusive, for $1 \leq i \leq j \leq |\outseq|$.
Let $\outsubseq{i}{i} = \epsilon$.

Define some indicator functions to match output substrings to rules
\begin{eqnarray*}
\Delta(i,j,x) & = & \delta \left( \outsubseq{i}{j}\ =\ x \right) \\
\Delta(i,x) & = & \Delta(i,i+|x|,x)
\end{eqnarray*}

\subsubsection{The dynamic programming matrix}

\newcommand\m{M}


{\bf more to go here}


\section{A simple retrotransposon grammar}
\seclabel{RetroGrammar}

An example of an observed indexed grammar is the following class of grammars that can generate context-free structures flanked by LTRs (long terminal repeats).

\subsection{SCFG and LTR components}

\subsection{Developing the SCFG sub-grammar for transposon contents}

\begin{itemize}
\item $X$ generates a nucleotide sampled from the background distribution
\item $X_L$ generates $\ell$ background nucleotides, where $\ell \sim L$
\item $F_N$ samples a DNA sequence coding for family $N$ from PFAM \cite{Pfam2008}
\item $I_L$ generates an intron of length $\ell \sim L$ (can be emitted by $F_N$)
\item $T_A$ generates a terminal inverted repeat, then transits to $A$
\end{itemize}
We can also make transitions back to $\startsymbol$ to generate a nested transposon insertion.

The grammar so described has some similarities to LTRdigest \cite{pmid19786494} and TENest \cite{KronmillerWise2008}.

Note a flaw of the framework is that, since the index string representing the ``consensus'' LTR must be directly observed at least once in the output
(in fact, it corresponds to the 5'-most repeat in this grammar), 
we cannot allow a nested transposon insertion {\em within} that LTR.


\subsection{Supplying external hints}

The parsing algorithm uses ${\cal O}(|\outseq|^4)$ memory and ${\cal O}(|\outseq|^4)$ time.
The hope is to accelerate it significantly by using externally-supplied ``hints''
as constraints on the locations of various features (especially the LTRs).

The hints file should include
\begin{itemize}
\item A set of tuples $(i,j,k,l)$ indicating that $\outsubseq{i}{j}$ and $\outsubseq{k}{l}$ are (respectively) the 5' and 3' repeat regions of an LTR
\item A set of tuples $(i,j,N)$ indicating that $\outsubseq{i}{j}$ is a match to PFAM family $N$
\end{itemize}

These hints can be generated by fast tools, e.g. suffix-tree based algorithms for finding LTRs \cite{pmid16819780},
or GeneWise for finding DNA sequences that code for PFAM protein domains \cite{BirneyEtAl04}.

A very quick heuristic that might achieve most of the benefits of a more rigorous ``hints'' constraint
would be to divide the genome into windows and only run the grammar on windows which contain $K$ or more of the appropriate hints.

\section{Glossary of mathematical notation}

\noindent
\begin{tabular}{ll}
Symbol & Meaning \\
\hline
$\grammar$ & Grammar \\
$\nodelabels$ & Set of node labels \\
$\terminals$ & Set of terminals \\
$\rules$ & Set of transformation rules \\
$\weight$ & Rule weight function \\
$\epsilon$ & The empty string \\
$\terminals^\ast$ & Set of strings over $\terminals$, including the empty string \\
$\outseq$ & Output sequence \\
$|\outseq|$ & Length of $\outseq$ \\
$\outsubseq{i}{j+1}$ & Substring of $\outseq$ from $i$ to $j$ inclusive ($i$ starts at 1) \\
$\outsubseq{i}{i}$ & The empty string \\
$\m(i,j,k,l,X)$ & Max parse tree weight for $\outsubseq{k}{l}$ rooted at $X[\outsubseq{i}{j}]$ \\
$\Delta(i,j,x)$ & Indicates if rule string $x$ matches output string $\outsubseq{i}{j}$ \\
$\Delta(i,x)$ & Indicates if rule string $x$ matches output string $\outseq$, starting at position $i$ \\
\end{tabular}

% Do NOT remove this, even if you are not including acknowledgments
\newpage
\section{Acknowledgments}

Our understanding of grammars applied to bioinformatics has been uplifted in conversations with
Sean Eddy, Mark Steedman, Bonnie Webber, Aravind Joshi, David Searls, Dan Klein, and Michael Souza.

\section{References}
% The bibtex filename(s)
\bibliography{../latex-inputs/alignment,../latex-inputs/reconstruction,../latex-inputs/duplication,../latex-inputs/genomics,../latex-inputs/ncrna,../latex-inputs/url,../latex-inputs/transposon}

%\clearpage
%\section{Figure Legends}

\end{document}

